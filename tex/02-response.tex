%! TEX root=../main.tex

\subsection{Response 2}
  \begin{quotation}
    Based on the Dictionary, non-verbal communication means psychol those
      aspects of communication, such as gestures and facial expressions, that
      do not involve verbal communication but which may include nonverbal
      aspects of speech itself (accent, tone of voice, speed of speaking, etc).
      Your non-verbal communication cues—the way you listen, look, move, and
      react—tell the person you’re communicating with whether or not you care,
      if you’re being truthful, and how well you’re listening. When your
      nonverbal signals match up with the words you’re saying, they increase
      trust, clarity, and rapport. When they don’t, they can generate tension,
      mistrust, and confusion. Have you ever been to a place that has a
      different language than what you can speak and can hear? That's what I
      experienced when I first arrived in the United States. English is not the
      everyday language I spoke when I was in Indonesia. When I first arrived in
      Denver, I was really interested in going to the museum but I was too shy
      to speak English, and I happened to be alone at that time. I was sure
      that I could speak good English, but I was very embarrassed at the time
      so what I did the first time I went to the museum was I showed my passport
      and my finger point. Very great they understand that I am a tourist from
      another country and do not understand English. From cynical I understand,
      without verbal language you can use non-verbal language using expressions,
      gestures, or objects that help you speak.

    The story was different when I was in the world of work in Indonesia. As we
      know getting to know people we are often introduced by someone else, but
      sometimes I feel I don't need to shake hands with them because I feel they
      don't need to know who I am, all I do is smile and bend my body a little
      as a sign of respect and the word   ``greetings'' non-verbally. Maybe this
      sounds rude because you don't want to shake hands with other people, but
      according to the Indonesian tradition, this is quite common for people to
      do in their daily activities and in the world of work.
  \end{quotation}

  \paragraph{This is a response to Marcella Ferchette on Post ID 43573269}
    Loved reading your post, Marcella. There is a really interesting point
      you had brought up in regards to non verbal communication across cultures
      which differ to extream degrees --- that they are not all
      \textit{completely} different! While many manerisms and cultural norms
      contrast when traveling between practically any country, there are still
      enough that stay the same to allow us a better chance at communicating
      when we do not speak the same language as those around us. I find that
      it is \textit{far} easier to learn another culture's body language and
      other non-verbal communication cues than it is to learn an entirely new
      language!
