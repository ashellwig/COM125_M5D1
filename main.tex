\documentclass[stu,12pt]{apa7}
  \usepackage{times}               % Times New Roman Font Face
  \usepackage[american]{babel}     % Localization
  \usepackage[utf8]{inputenc}      % Input Encoding
  \usepackage{hyperref}            % Hyperlinks
  \usepackage{enumitem}            % Additional Enumeration Environment Settings
  \usepackage{geometry}            % Page Layout
  \usepackage{soul}                % Text Highlighting
  \usepackage{graphicx}            % Images
  \usepackage{csquotes}            % Quoting Environment
  \usepackage{bookmark}            % Required by `csquotes'
  \usepackage{mdframed}            % Colorful Tex-Box Environment
  \usepackage[toc]{appendix}       % Appendix
  \usepackage{fancyhdr}            % Headings and Footers
  \usepackage[%
    style=apa,%
    sortcites=true,%
    sorting=nyt%
  ]{biblatex}
  \usepackage{xcolor}

  % Bibliography Setup
  %% Language Mappings
  \DeclareLanguageMapping{english}{english-apa}
  \DeclareLanguageMapping{american}{american-apa}
  %% Bibliography File Path
  \addbibresource{main.bib}
  %% Categories for Specified Bibliography Items
  %%% Category for sources not referenced in-text
  \DeclareBibliographyCategory{consulted}
  \addtocategory{consulted}{noauthor_business_nodate}
  \addtocategory{consulted}{noauthor_communication_2013}
  \addtocategory{consulted}{noauthor_college_nodate}

  % Hyperlink Setup
  \hypersetup{
    colorlinks = true,
    urlcolor = blue,
    linkcolor = blue,
    citecolor = blue
  }

  % Page and Text Layout
  \geometry{%
    a4paper,%
    top=1in,%
    bottom=1in,%
    left=1in,%
    right=1in%
  }
  \setlength{\headheight}{15pt}


  % Title Page
  \title{%
    M5D1: Video Scavenger Hunt
  }
  \shorttitle{Module 5 Discussion 1}
  \author{Ashton Hellwig}
  \authorsaffiliations{Department of Mathematics, Front Range Community College}
  \course{COM125: Interpersonal Communication}
  \professor{Richard Thomas}
  \duedate{December 12, 2020 23:59:59 MDT}
  \date{\today}
  \lhead{COM125CG1-M5D1}
  \abstract{%
    \textbf{Overview}\\%
    Throughout the course, you have watched clips from movies that illustrate
      examples of interpersonal communication.\\%

    Now it’s your turn to find video clips that show the types of challenges
      you have been working on during this class!\\%

    You should spend approximately 4 hours on this assignment.%
  }

\begin{document}
  % Title Page
  \maketitle

  % Initial Post
  \section{Initial Post}
    \subsection*{Instructions}
      \begin{enumerate}
        \item Select three (3) interpersonal communications skills that you
          worked to improve during this course. Two of the three interpersonal
          communication skills may comprise the ones you practiced and added to
          the Word document each week. At least one should be a skill you feel
          did not improve, or feel you improved less than the others.
        \item Search the internet for video clips (2--5 minutes long)
          illustrating those three skills.
        \item In your written post, list the three skills you selected. For
          each video clip, describe the interaction that is taking place in the
          video clip and identify the interpersonal communication skills that
          are demonstrated. Explain how the people in the clip were using the
          skills, and identify what you can learn from them that might help you
          work on those skills in the future.
        \item Include links to the video clips in your post. Refrain from using
          any video clips that were previously used in the course.
      \end{enumerate}


    \newpage
    \subsection{Skills I Have Improved On}
      \subsubsection{Skill One: Grasping Psychological Context}
        Grasping psychological context was originally something at sort of a
          higher level of difficulty for me than many other ideas within the
          realm of ``\textit{interpersonal communication}''. More recently, I
          have done my best to get a thorough read of the environmental aspects
          in addition to the ``aura'' --- for lack of a better term --- of the
          room prior to saying whatever was initially on my mind prior to
          entering the conversation.

        \subsubsection*{Video Illustrating Skill 1}
          What follows is a video link illustrating \textit{an understanding
            of psychological context}.

          \href{https://www.youtube.com/watch?v=cSSB4qPrfOg}{Adventure Time: %
            Ocean of Fear}.

          In the previously mentioned video clip, we have two subjects: Finn
            (the human) and Jake (the dog) of Adventure Time. In the clip, we
            can see how well jake does at managing the stress of his friend
            Finn with conquering his fear of the ocean
            \parencite[00:00:30--00:01:00]{leichliter_adventure_2017}.


      \subsubsection{Skill Two: Empathy}
        Empathy is one of the most important aspects, in this author's opinion,
          in regards to interpersonal communication. Empathy, to me, provides
          the deepest expression of understanding available to the other
          individuals involved in the conversation as the emulation of emotions
          shows that you are feeling the same pain as the speaker is when
          discussing a situation with them.

        \subsubsection*{Video Illustration of Empathy}
          What follows is a video link illustrating \textit{Empathy}.

          \href{https://www.youtube.com/watch?v=AuD_VzYku78}{Adventure Time:
            Normal Face}.

          In the previous scene, we see Jake taking the funeral of a peer with
            an incredibly complex character development and strange relationship
            to our main characters. Finn illustrates empathy when he enters the
            supply closet in an attempt to cheer up his friend Jake. Here,
            Finn was able to maintain his ``normal face'' without crying until
            he empathized with Jake to the extent of losing his grip on his own
            emotions when seeing Jake cry
            \parencite{leichliter_adventure_2015}.


    \subsection{The One I am Still Working On}
      \subsubsection{Listening with The Intent to Understand, Not Reply}
        Listening with the intent to understand, rather than with the intent to
          respond, is an incredibly important skill I had felt I lacked in
          my interpersonal communication skill chest for a significant part of
          my adolescent and adult life. Before I was conscience of this fact, I
          would immediately relate what someone was telling me is happening in
          their life to myself or how my experience could help them understand
          or get through their own. Unfortunately, I still have this habit
          (as do my peers I shared this assignment with who also feel this
          skill is \textit{incredibly} difficult to master).

        \subsubsection*{Video Illustrating Skill 3}
          What follows is a video link illustrating \textit{listening with the
            intent to understand, rather than with the intent to respond}.

          \href{https://youtu.be/sk_M0-3qaVQ?t=102}%
            {Adventure Time: Burning Low}.

          In the previous scene we see Finn speaking with Princess Bubblegum
            about her concerns with him dating another princess on the planet,
            named Flame Princess. Finn did not listen well enough to understand
            that it is not that Princess Bubblegum is romantically interested in
            him being the reason he cannot see her, but rather that ``love''
            would cause some sort of reaction too dangerous for them to handle
            \parencite[00:01:02--00:01:57]{leichliter_adventure_2018}.


  % Replies
  % %! TEX root=../main.tex

\section{Responses}
  \subsection{Response 1}
    \begin{quotation}
      The first thing I've begun to work on in my communication is clarity in
        speech. In this movie clip it shows Ace Ventura speaking very quickly
        while keeping his speech clear. It shows multiple scenes where he takes
        a deep breath and quickly answers their questions fully while not
        stuttering or making a mistake. He stays confident in his tone. One
        thing I can learn from this is to remember to take a deep breath. This
        helps you to stay calm so that you can stay clear in your speech.

      Another thing I have been working on is managing emotions in speech. In
        this video clip Peter goes through a couple of things that for a lot of
        people would have caused an emotional outbreak. For one, he was talked
        to by both bosses when neither of them seemed to listen to him, and
        another coworker was playing the radio which seemed to annoy Peter.
        With all of these things, Peter could have reacted with anger or
        frustration. Instead, he stayed calm and moved on. He didn't allow the
        things to bother him too much and moved on which is something I could
        learn to do.

      The final thing I could have worked more on was nonverbal communication
        skills. This was something I didn't focus enough on. In the first scene
        of this clip, Lucy struggles with her nonverbal communication skills
        but after each try gets a little better. She's working on her facial
        expressions and by smiling is able to conceal her emotions a little.
        Something I learned from this is it can be difficult sometimes, but it
        helps to try and smile through it.
    \end{quotation}

    \paragraph{This is a response to Julayne Kilcullen on Post ID 43614390}
      Placeholder.

  % %! TEX root=../main.tex

\subsection{Response 2}
  \begin{quotation}
    Placeholder.
  \end{quotation}

  \paragraph{This is a response to FIRST LAST on Post ID 00000000}
    Placeholder.



  % Bibliography
  %% Works Cited
  \newpage
  \printbibliography[%
    title={References},%
    heading={bibintoc},%
    notcategory={consulted}%
  ]

  %% Works Consulted
  \newpage
  \nocite{*}
  \printbibliography[%
    title={Additional References},%
    heading={bibintoc},%
    category={consulted}%
  ]
\end{document}
